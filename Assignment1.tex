\documentclass[journal,12pt,twocolumn]{IEEEtran}

\usepackage{setspace}
\usepackage{gensymb}
\singlespacing
\usepackage[cmex10]{amsmath}

\usepackage{amsthm}

\usepackage{mathrsfs}
\usepackage{txfonts}
\usepackage{stfloats}
\usepackage{bm}
\usepackage{cite}
\usepackage{cases}
\usepackage{subfig}

\usepackage{longtable}
\usepackage{multirow}

\usepackage{enumitem}
\usepackage{mathtools}
\usepackage{steinmetz}
\usepackage{tikz}
\usepackage{circuitikz}
\usepackage{verbatim}
\usepackage{tfrupee}
\usepackage[breaklinks=true]{hyperref}
\usepackage{graphicx}
\usepackage{tkz-euclide}

\usetikzlibrary{calc,math}
\usepackage{listings}
    \usepackage{color}                                            %%
    \usepackage{array}                                            %%
    \usepackage{longtable}                                        %%
    \usepackage{calc}                                             %%
    \usepackage{multirow}                                         %%
    \usepackage{hhline}                                           %%
    \usepackage{ifthen}                                           %%
    \usepackage{lscape}     
\usepackage{multicol}
\usepackage{chngcntr}

\DeclareMathOperator*{\Res}{Res}

\renewcommand\thesection{\arabic{section}}
\renewcommand\thesubsection{\thesection.\arabic{subsection}}
\renewcommand\thesubsubsection{\thesubsection.\arabic{subsubsection}}

\renewcommand\thesectiondis{\arabic{section}}
\renewcommand\thesubsectiondis{\thesectiondis.\arabic{subsection}}
\renewcommand\thesubsubsectiondis{\thesubsectiondis.\arabic{subsubsection}}


\hyphenation{op-tical net-works semi-conduc-tor}
\def\inputGnumericTable{}                                 %%

\lstset{
%language=C,
frame=single, 
breaklines=true,
columns=fullflexible
}
\begin{document}


\newtheorem{theorem}{Theorem}[section]
\newtheorem{problem}{Problem}
\newtheorem{proposition}{Proposition}[section]
\newtheorem{lemma}{Lemma}[section]
\newtheorem{corollary}[theorem]{Corollary}
\newtheorem{example}{Example}[section]
\newtheorem{definition}[problem]{Definition}

\newcommand{\BEQA}{\begin{eqnarray}}
\newcommand{\EEQA}{\end{eqnarray}}
\newcommand{\define}{\stackrel{\triangle}{=}}
\bibliographystyle{IEEEtran}
\raggedbottom
\setlength{\parindent}{0pt}
\providecommand{\mbf}{\mathbf}
\providecommand{\pr}[1]{\ensuremath{\Pr\left(#1\right)}}
\providecommand{\qfunc}[1]{\ensuremath{Q\left(#1\right)}}
\providecommand{\sbrak}[1]{\ensuremath{{}\left[#1\right]}}
\providecommand{\lsbrak}[1]{\ensuremath{{}\left[#1\right.}}
\providecommand{\rsbrak}[1]{\ensuremath{{}\left.#1\right]}}
\providecommand{\brak}[1]{\ensuremath{\left(#1\right)}}
\providecommand{\lbrak}[1]{\ensuremath{\left(#1\right.}}
\providecommand{\rbrak}[1]{\ensuremath{\left.#1\right)}}
\providecommand{\cbrak}[1]{\ensuremath{\left\{#1\right\}}}
\providecommand{\lcbrak}[1]{\ensuremath{\left\{#1\right.}}
\providecommand{\rcbrak}[1]{\ensuremath{\left.#1\right\}}}
\theoremstyle{remark}
\newtheorem{rem}{Remark}
\newcommand{\sgn}{\mathop{\mathrm{sgn}}}
\providecommand{\abs}[1]{\left\vert#1\right\vert}
\providecommand{\res}[1]{\Res\displaylimits_{#1}} 
\providecommand{\norm}[1]{\left\lVert#1\right\rVert}
%\providecommand{\norm}[1]{\lVert#1\rVert}
\providecommand{\mtx}[1]{\mathbf{#1}}
\providecommand{\mean}[1]{E\left[ #1 \right]}
\providecommand{\fourier}{\overset{\mathcal{F}}{ \rightleftharpoons}}
%\providecommand{\hilbert}{\overset{\mathcal{H}}{ \rightleftharpoons}}
\providecommand{\system}{\overset{\mathcal{H}}{ \longleftrightarrow}}
	%\newcommand{\solution}[2]{\textbf{Solution:}{#1}}
\newcommand{\solution}{\noindent \textbf{Solution: }}
\newcommand{\cosec}{\,\text{cosec}\,}
\providecommand{\dec}[2]{\ensuremath{\overset{#1}{\underset{#2}{\gtrless}}}}
\newcommand{\myvec}[1]{\ensuremath{\begin{pmatrix}#1\end{pmatrix}}}
\newcommand{\mydet}[1]{\ensuremath{\begin{vmatrix}#1\end{vmatrix}}}
\numberwithin{equation}{subsection}
\makeatletter
\@addtoreset{figure}{problem}
\makeatother
\let\StandardTheFigure\thefigure
\let\vec\mathbf
\renewcommand{\thefigure}{\theproblem}
\def\putbox#1#2#3{\makebox[0in][l]{\makebox[#1][l]{}\raisebox{\baselineskip}[0in][0in]{\raisebox{#2}[0in][0in]{#3}}}}
     \def\rightbox#1{\makebox[0in][r]{#1}}
     \def\centbox#1{\makebox[0in]{#1}}
     \def\topbox#1{\raisebox{-\baselineskip}[0in][0in]{#1}}
     \def\midbox#1{\raisebox{-0.5\baselineskip}[0in][0in]{#1}}
\vspace{3cm}
\title{Assignment 1}
\author{P Ganesh Nikhil Madhav -CS20BTECH11036}
\maketitle
\newpage
\bigskip
\renewcommand{\thefigure}{\theenumi}
\renewcommand{\thetable}{\theenumi}
Download all python codes from 
\begin{lstlisting}
https://github.com/Nik123-cpp/Assignment-1/blob/main/assignment1.py
\end{lstlisting}
%
and latex-tikz codes from 
%
\begin{lstlisting}
https://github.com/Nik123-cpp/Assignment-1/blob/main/Assignment1.tex
\end{lstlisting}
\section{Problem 3.4}
The probability that a bulb produced by a factory will fuse after 150 days is 0.05.Find the probability that out of 5 such bulbs\\ (i) none \\(ii) not more than one\\(iii)more than one\\(iv) atleast one\\ will fuse after 150 days of use.\\\\
\textbf{Solution}\\
Let X be random variable which denoting number of bulbs fuses after 150 days of use,among the 5 bulbs.Then by Binomial Distribution.
\begin{align}
    \pr{X=k}&=\binom{n}{k} p^k\brak{1-p}^{n-k}
    \\
    k=0,\dots,n
    \label{eq:exam41_1}
\end{align}
For given question n = 5 , p = 0.05,1-p = 0.95.
\begin{enumerate}
\item From \eqref{eq:exam41_1}
\begin{align}
    \pr{X=0}&=\binom{5}{0}(0.05)^0\brak{0.95}^{5}
    =0.77378094 
    \label{eq:exam41_2}
\end{align}
i.e the probability of all 5 bulbs working after 150 days of use
\item simillarly
\begin{align}
    \pr{X\leq1}&=\sum_{k=0}^{1}\pr{X=k}
    \\
    &=\sum_{K=0}^{1}\binom{5}{k} (0.050^k\brak{0.95}^{5-k}
    \\
    &=0.9774075025
    \label{eq:exam41_3}
\end{align}
i.e the probabity of either none of the bulbs or exactly one bulb will fuse after 150 days of use of 5 such bulbs.
\item
\begin{align}
    \pr{X>1}&=\sum_{k=2}^{5}\pr{X=k}
\end{align}
Which is complement of second case i.e case(ii),So we can write 
\begin{align}
    \pr{X>1}&= 1-\pr{X\leq1}
\end{align}
From \eqref{eq:exam41_3}
\begin{align}
    \pr{X>1}&= 1-0.9774075025
    \\
    &=0.0225924975
\end{align}
\item
\begin{align}
    \pr{X\geq1}=\sum_{k=1}^{5}\pr{X=k}
\end{align}
which is complement of first case i.e case(i),So we can write 
\begin{align}
    \pr{X\geq1}&= 1-\pr{X<1}
\end{align}
From \eqref{eq:exam41_2}
\begin{align}
    \pr{X\geq1}&= 1-0.77378094
    \\
    &=0.22621906   
\end{align}
\newpage Below  table is indicating probability of various events to occur\\
\begin{center}
 \begin{tabular}{||c|| c ||} 
 \hline
 If Number of bulbs fuses after 150days of use among 5 bulbs
 is & Then Probability for the event is\\ [0.5ex] 
 \hline\hline
 none &  0.77378094\\ 
 \hline
 not more than 1 & 0.977407502 \\
 \hline
 more than 1 & 0.022592497 \\
 \hline
 atleast 1 & 0.22621906 \\
 \hline
\end{tabular}
\end{center}
\end{enumerate}
\end{document}
